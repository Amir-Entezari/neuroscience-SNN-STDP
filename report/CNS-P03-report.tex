\documentclass{report}

\usepackage[textwidth=16cm, textheight=24cm]{geometry}
% Packages for special characters and symbols
\usepackage[utf8]{inputenc}
\usepackage[T1]{fontenc}
\usepackage{amsmath, amssymb, amsthm}

% Package for hyperlinks
\usepackage{hyperref}

% Package for graphics
\usepackage{graphicx}
\usepackage{float}
\usepackage{listings}
\usepackage{dirtree}

% Package for better tables
\usepackage{booktabs}
\usepackage{xepersian}

% \setlatintextfont{Times New Roman}
% \settextfont{Yas.ttf}
\settextfont[
BoldFont=Yas Bd.TTF]{Yas.TTF}
% Set the title, author, and date
\title{گزارش پروژه دوم علوم اعصاب محاسباتی}
\author{امیرحسین انتظاری}
\date{\today}

% Begin document
\begin{document}

% Create the title
\maketitle
\newpage
% Create a table of contents
\tableofcontents
% Separate the table of contents from the next content with a new page
% \clearpage

\newgeometry{textwidth=15cm,textheight=22cm}
    \begin{abstract}
        هدف از این پروژه، پیاده سازی کدگذاری کردن ورودی شبکه، یادگیری
        (بدون ناظر و تقویتی) 
        است. در این پروژه از مباحثی که در پروژه های قبلی یاد گرفتیم،
        ( مانند مدل های نورونی، سیناپس و...)، 
        استفاده می‌کنیم.
    \end{abstract}
\restoregeometry


\newpage
\section{پیاده سازی کدگذاری ها}
    در این بخش می‌خواهیم نحوه کد کردن اطلاعات
    (مانند یک تصویر) 
    در شبکه های عصبی ضربه‌ای را پیاده سازی و تحلیل کنیم. برای اینکار، روش های مختلفی وجود دارد، مانند روش 
    \lr{time-to-first-spike}، 
    کدگذاری ترتیبی
    \footnote{\lr{Rank-order encoding}}، 
    کدگذاری براساس تاخیر
    \footnote{\lr{Latency encoding}}، 
    کدگذاری براساس همزمانی ضربه ها
    \footnote{\lr{Coding by synchrony}}، 
    کدگذاری اعداد، کدگذاری به روش پواسون
    \footnote{Poisson} 
    و روش های دیگر بسیار که کاربرد های مختلفی دارند. 
    در این پروژه، ما تمرکزمان را روی سه روش کدگذاری زیر میگذاریم و آن ها پیاده سازی و تحلیل میکنیم:
    \begin{itemize}
        \item روش کدگذاری 
        \lr{time-to-first-spike}
        \item روش کدگذاری مقادیر عددی
        \item روش کدگذاری به کمک توزیع پواسون
    \end{itemize}

    \subsection*{کدگذاری به روش 
    \lr{time-to-first-spike}}
        
\newpage

\section{یادگیری بدون ناظر}
    در مطالعه علوم اعصاب محاسباتی، درک مکانیسم‌های پلاستیسیته سیناپسی
    \footnote{\lr{synaptic plasticity}}
    - چگونگی تقویت یا تضعیف ارتباطات بین نورون‌ها در طول زمان، که از آن به عنوان یادگیری نیز یاد می‌شود - بسیار مهم است. یکی از مدل‌های اصلی برای شبیه‌سازی یادگیری در مغز، قانون یادگیری انعطاف‌پذیری وابسته به زمان ضربه 
    ($STDP$ \footnote{Spike-Timing-Dependent Plasticity})
    است. این مدل به عنوان یک چارچوب ساده ولی قوی برای بررسی چگونگی تأثیر زمان ضربه ها در ارتباطات عصبی بر ارتباطات سیناپسی، در نتیجه یادگیری و حافظه در شبکه‌های عصبی مغز قرار می‌گیرد.

    $STDP$
    یک فرآیند بیولوژیکی است که قدرت اتصالات بین نورون های مغز را تنظیم می کند. این فرآیند، نقاط قوت اتصال را بر اساس زمان‌بندی نسبی خروجی یک نورون خاص و پتانسیل‌های عمل ورودی 
    (یا ضربه)
    تنظیم می‌کند.

    در فرآیند 
    $STDP،$، 
    اگر یک ضربه ورودی به یک نورون، به طور متوسط، بلافاصله قبل از ضربه خروجی آن نورون رخ دهد، آن ورودی خاص تا حدودی قوی‌تر می‌شود. اگر یک ضربه ورودی، به طور متوسط، بلافاصله پس از یک ضربه خروجی رخ دهد، آن ورودی خاص تا حدودی ضعیف‌تر می‌شود، بنابراین معنای «یادگیری انعطاف‌پذیری وابسته به زمان ضربه» برای ما روشن تر می شود. بنابراین، ورودی‌هایی که ممکن است علت برانگیختگی نورون پس سیناپسی باشند، احتمالاً در آینده کمک می‌کنند، در حالی که ورودی‌هایی که علت ضربه پس سیناپسی نیستند، کمتر در آینده کمک می‌کنند. این فرآیند تا زمانی ادامه می‌یابد که زیرمجموعه‌ای از مجموعه اولیه اتصالات باقی می‌ماند، در حالی که تأثیر همه اتصالات دیگر به 0 کاهش می‌یابد.
    \cite{STDP-Wikipedia}
    از آنجایی که یک نورون زمانی که بسیاری از ورودی‌هایش در یک دوره کوتاه اتفاق می‌افتند یک ضربه خروجی تولید می‌کند، زیرمجموعه ورودی‌هایی که باقی می‌مانند عبارتند از آنهایی که تمایل به همبستگی در زمان داشتند. علاوه بر این، از آنجایی که ورودی‌هایی که قبل از خروجی تقویت می‌شوند، ورودی‌هایی که اولین نشانه‌های همبستگی را ارائه می‌دهند، در نهایت به ورودی نهایی نورون تبدیل می‌شوند.

    $STDP$ 
    خود به چندین مدل میتواند پیاده سازی شود که از بین این مدل ها، مدل 
    \lr{Pair-based STDP wtih local variables}
    برای بررسی انتخاب شده است.

    \subsection{\lr{Pair-based STDP}}
        در مدل 
        $STDP$ 
        مبتنی بر جفت، تنظیم وزن های سیناپسی با لحظه‌های دقیق وقوع ضربه در نورون‌های پیش سیناپسی و پس سیناپسی انجام می‌شود. این مدل از دو متغیر محلی - اثر
        \footnote{trace}
        ضربه پیش سیناپسی و اثر ضربه پس سیناپسی - برای ثبت تاریخچه ضربه عصبی استفاده می کند که بر تغییرات سیناپسی تأثیر می گذارد.

        \begin{itemize}
            \item \textbf{اثر ضربه پیش سیناپسی $x_j$:}
                این اثر در طول زمان با یک ثابت کاهشی، کاهش می‌یابد اما با هر ضربه پیش سیناپسی دوباره افزایش می‌یابد و به طور موثر زمان این ضربه ها را مشخص می‌کند. نحوه بروزرسانی این مقدار توسط فرمول زیر می تواند محاسبه شود:
                \begin{equation}
                    \frac{dx_j}{dt} = -\frac{x_j}{\tau_+} + \sum_{f}\delta(t-t_{j}^{f})
                \end{equation}
            \item \textbf{اثر ضربه پس سیناپسی $y_i$:}
            مشابه اثر پیش سیناپسی، این اثر نیز در طول زمان تحلیل می‌رود، اما با ضربه پس سیناپسی افزایش می‌یابد.  نحوه بروزرسانی این مقدار توسط فرمول زیر می تواند محاسبه شود:
            \begin{equation}
                \frac{dy_i}{dt} = -\frac{y_i}{\tau_-} + \sum_{f}\delta(t-t_{i}^{f})
            \end{equation}
        \end{itemize}

        این اثر ها
        (ردپاها)
        بسیار مهم هستند زیرا مبنایی برای محاسبه تغییرات سیناپسی بلافاصله پس از ضربه زدن هستند و باعث می‌شود که مدل به زمان ضربه پاسخ دهد.

        سپس نوبت به تغییرات وزن ها میرسد. این تغییرات بر اساس مقادیر فعلی اثر های گفته شده در هر زمان که یک ضربه زده می شود محاسبه می شود. این روش نه تنها با مشاهدات بیولوژیکی هماهنگ است، بلکه امکان شبیه سازی دقیق فرآیندهای یادگیری و تشکیل حافظه در شبکه را نیز فراهم می کند. نحوه بروزرسانی وزن ها نیز توسط فرمول زیر امکان پذیر است:
        \begin{equation}
            \frac{dw_{ij}}{dt} = -A_{-}(w_{ij})y_{i}(t)\sum_{f}\delta(t-t^{f}_{j})+A_{+}(w_{ij})x_{j}(t)\sum_{f}\delta(t-t^{f}_{i})
        \end{equation}

        قبل از رفتن به آزمایش کردن مدل، یک نکته دیگر باقی می ماند و آن این است که برای 
        $A_{-}$، $A_{+}$ 
        میتواند توابعی
        \footnote{\lr{Soft/Hard bound}}
        برای کنترل کردن محدوده وزن ها استفاده شود.
        
        
    \subsection{آزمایش ها}
        \paragraph*{شباهت کسینوسی:}
            \textbf{ دقت شود که نمودار شباهت کسینوسی در تمام شکل ها آمده است.}

        حال در این قسمت، به سراغ آزمایش مدل 
        $STDP$ 
        مان میرویم و نمودار های خواسته شده در موارد الف تا و را بررسی میکنیم. 
        
        ابتدا دو الگوی به صورت آرایه را به مدل میدهیم. برای اینکار میتوانیم یک آرایه تصادفی از اعداد ۱ تا ۱۰ و یک آرایه تصادفی دیگر که مقادیر آن دوبرابر آرایه اول است را به عنوان ورودی به جمعیت ورودی بدهیم. از این رو، مطابق شکل 
        \ref{fig:part2-small-array-stdp}
        مشاهده میکنیم که با تغییر وزن های سیناپسی، هر نورون یکی از الگو ها را یادگرفته است. به طوری که وقتی الگوی یاد گرفته شده را میبیند نرخ ضربه زدن آن بیشتر می شود.
        \begin{figure}[!ht]
            \centering
            \includegraphics[width=1\textwidth]{plots/part2-small-array-stdp.pdf} 
            \caption{قانون یادگیری 
            $STDP$.
            همانطور که از شکل نیز مشخص است، وزن های سیناپسی طوری تنظیم می شوند که هر نورون، یکی از الگو ها را یاد میگیرد. همچنین طبق نمودار شباهت کسینوسی، به مرور زمان، شباهت بین وزن های دونورون خروجی نیز کاهش می یابد.}
            \label{fig:part2-small-array-stdp}
        \end{figure}

        \subsubsection*{تغییر در اندازه ورودی}
        هر چند این یادگیری به دلیل بدون ناظر بودن شبکه و همچنین ساده بودن آن، از عهده یادگیری الگو های پیچیده تر ممکن است برنیاید. به طور مثال، با زیاد کردن طول آرایه، ویژگی های مورد نیاز برای تشخیص دادن نیز افزایش یافته پیداست، یادگیری ممکن است به خوبی انجام نشود. به عنوان مثال، در شکل 
        \ref{fig:part2-large-array-stdp}
        یادگیری به خوبی انجام شده است، ولی نسبت به شکل 
        \ref{fig:part2-small-array-stdp} 
        دیرتر یادگیری شکل گرفته است.

        \begin{figure}[htbp]
            \centering
            \includegraphics[width=1\textwidth]{plots/part2-large-array-stdp.pdf} 
            \caption{قانون یادگیری 
            $STDP$ برای ورودی بزرگتر.
            مطابق شکل ملاحظه میکنیم علی رغم اینکه مدل توانسته تا تکرار ۱۰۰۰ ام شبیه سازی به خوبی الگو ها را یاد بگیرد، ولی نسبت به شکل
            \ref{fig:part2-small-array-stdp}
            که ورودی کوچکتر بود این اتفاق دیرتر افتاده است. هرچند با تغییر پارامتر هایی مانند پنجره زمانی یا نرخ یادگیری می‌توان این یادگیری را سریع تر انجام داد که این موضوع را در ادامه بررسی خواهیم کرد.}
            \label{fig:part2-large-array-stdp}
        \end{figure}
        
        حال مطابق خواسته پروژه، ورودی ها را به گونه‌ای به شبکه می‌دهیم که نورون های ورودی در دریافت محرک هم‌پوشانی داشته باشند. یعنی بعضی نورون ها، مسئول گرفتن هر دو ورودی باشند. 
        (مطابق شکل \ref{fig:part2-input-overlap})
        \begin{figure}[!ht]
            \centering
            \captionsetup{width=.9\linewidth}
            \includegraphics[width=0.9\textwidth]{images/input-overlap.png} 
            \caption{نورون هایی که با رنگ سبز مشخص شده اند، هر دو الگو را به عنوان ورودی دریافت میکنند.
            }
            \label{fig:part2-input-overlap}
        \end{figure}

        ابتدا با هم‌پوشانی کم شروع می‌کنیم. مطابق شکل
        \ref{fig:part2-array-stdp-overlap}
        دریافت می شود که هم‌پوشانی نورون ها در دریافت ورودی نمی‌تواند مانع یادگرفتن مدل شود هر چند که سرعت آن را کاهش می‌دهد.
        \begin{figure}[!ht]
            \centering
            \captionsetup{width=.9\linewidth}
            \includegraphics[width=0.9\textwidth]{plots/part2-array-stdp-overlap.pdf} 
            \caption{هم‌پوشانی نسبی نورون های لایه اول برای دریافت ورودی. مطابق شکل ملاحظه میکنیم که هم‌پوشانی بین نورون های ورودی باعث می‌شود یادگیری در زمان کمی دیرتری اتفاق بیوفتد ولی مانع آن نمی‌شود.
            }
            \label{fig:part2-array-stdp-overlap}
        \end{figure}

        حال این هم‌پوشانی را حداکثر میکنیم به طوری که تمام نورون های ورودی مسئول دریافت هر دو الگو باشند. انتظار داریم که افزایش این هم‌پوشانی، منجر به اختلال در یادگیری شود یا یادگیری را به تعویق بیندازد. شکل
        \ref{fig:part2-array-stdp-high-overlap}
        این موضوع را تایید میکند.

        \begin{figure}[!ht]
            \centering
            \captionsetup{width=.9\linewidth}
            \includegraphics[width=0.9\textwidth]{plots/part2-array-stdp-high-overlap.pdf} 
            \caption{هم‌پوشانی کامل نورون های لایه اول برای دریافت ورودی. مطابق شکل ملاحظه میکنیم که هم‌پوشانی کامل بین نورون های ورودی باعث می‌شود یادگیری مختل شود و مثلا در شکل بالا فقط یکی از نورون ها یکی از الگو ها را یاد بگیرد. این به دلیل ساده بودن مدل 
            $STDP$ 
            است. به طوری که در بخش سوم مشاهده میکنیم که مدل 
            $RSTDP$ 
            نسبت به این هم‌پوشانی پایدار است.
            }
            \label{fig:part2-array-stdp-high-overlap}
        \end{figure}
        ممکن است این موضوع که مدلمان می تواند الگو های نظیر آرایه های بالا را تشخیص دهد، کنجکاوی ما را برانگیزد تا بخواهیم الگو های پیچیده تری مانند تصویر را نیز تشخیص دهیم. برای اینکار، ابتدا از الگو های ساده تری مانند تصاویر 
        \ref{fig:part2-pattern1} و 
        \ref{fig:part2-pattern2} 
        استفاده میکنیم تا قدرت مدل را بسنجیم و سپس به سراغ الگو های پیچیده تر می رویم.

        \begin{figure}[!ht]
            \centering
            \captionsetup{width=.9\linewidth}
            \begin{subfigure}[b]{0.35\textwidth}
                \centering
                \includegraphics[width=\textwidth]{images/pattern1.png}
                \caption{الگوی اول}
                \label{fig:part2-pattern1}
            \end{subfigure}
            \hfill
            \begin{subfigure}[b]{0.35\textwidth}
                \centering
                \includegraphics[width=\textwidth]{images/pattern2.png}
                \caption{الگوی دوم}
                \label{fig:part2-pattern2}
            \end{subfigure}
            \caption{دو تصویر به عنوان محرک}
            \label{fig:patterns}
        \end{figure}

        مطابق شکل 
        \ref{fig:part2-pattern-stdp} 
        ملاحظه میکنیم که مدل به خوبی از عهده تشخیص الگو های ساده 
        \ref{fig:part2-pattern1} و 
        \ref{fig:part2-pattern2} 
        برآمد و میتواند آن ها را تشخیص دهد.

        \begin{figure}[htbp]
            \centering
            \includegraphics[width=1\textwidth]{plots/part2-pattern-stdp.pdf} 
            \caption{قانون یادگیری 
            $STDP$ برای ورودی تصویر ساده. 
            مشاهده میکنیم که مدل توانسته به خوبی پ. الگوی تصویری ساده را از یکدیگر تشخیص دهد.
            }
            \label{fig:part2-pattern-stdp}
        \end{figure}

        حال یک قدم فراتر رفته و یک تصویر بزرگتر را به عنوان ورودی به مدل می‌دهیم. ابعاد این تصاویر بزرگتر بوده
        ($10\times 10$) 
        و پیچیده تر هستند. 
        (تصاویر 
        \ref{fig:part2-input-image1} و
        \ref{fig:part2-input-image2})
        \begin{figure}[htbp]
            \centering
            \captionsetup{width=.9\linewidth}
            \begin{subfigure}[b]{0.35\textwidth}
                \centering
                \includegraphics[width=\textwidth]{images/slop-resized.jpg}
                \caption{تصویر اول}
                \label{fig:part2-input-image1}
            \end{subfigure}
            \hfill
            \begin{subfigure}[b]{0.35\textwidth}
                \centering
                \includegraphics[width=\textwidth]{images/bird-resized.jpg}
                \caption{تصویر دوم}
                \label{fig:part2-input-image2}
            \end{subfigure}
            \caption{یک تصویر به عنوان محرک}
            \label{fig:part2-input-images}
        \end{figure}

        هرچند انتظار زیادی از مدل 
        $STDP$ 
        ساده برای تشخیص تصاویر نداریم، ولی مطابق شکل 
        \ref{fig:part2-images-stdp}
        به نظر می‌رسد که مدل پس از تکرار های بیشتری نسبت به حالت های قبل، می تواند از عهده تشخیص این الگو ها نیز برآید!
        \begin{figure}[htbp]
            \centering
            \includegraphics[width=1\textwidth]{plots/part2-images-stdp.pdf} 
            \caption{قانون یادگیری 
            $STDP$ برای ورودی تصویر. 
            مطابق شکل دریافت می‌شود که مدل توانسته است دو ورودی تصویر را نیز از یکدیگر تشخیص دهد.
            }
            \label{fig:part2-images-stdp}
        \end{figure}
        
        مشابه کاری که برای آرایه ها کردیم، اینجا نیز هم‌پوشانی نورون های ورودی را برای دریافت محرک بیشتر میکنیم تا تاثیر آن برایمان روشن تر شود.
        (شکل \ref{fig:part2-images-stdp-overlap})
        \begin{figure}[htbp]
            \centering
            \captionsetup{width=.9\linewidth}
            \begin{subfigure}[b]{0.9\textwidth}
                \centering
                \includegraphics[width=\textwidth]{plots/part2-images-stdp-overlap.pdf}
                \caption{ورودی با هم‌پوشانی نسبی.}
                \label{fig:part2-images-stdp-norm-overlap}
            \end{subfigure}
            \begin{subfigure}[b]{0.9\textwidth}
                \centering
                \includegraphics[width=\textwidth]{plots/part2-images-stdp-high-overlap.pdf}
                \caption{ورودی با هم‌پوشانی کامل.}
                \label{fig:part2-images-stdp-high-overlap}
            \end{subfigure}
            \caption{مشاهده میکنیم که افزایش هم‌پوشانی همانند آرایه ها باعث می‌شود یادگیری نتواند به طور صحیح انجام شود.}
            \label{fig:part2-images-stdp-overlap}
        \end{figure}

        \clearpage
        \subsubsection*{تغییر در وزن های اولیه}
            در شبکه‌های عصبی مصنوعی، انتخاب وزن‌های اولیه در شکل‌دهی مسیر یادگیری و عملکرد کلی شبکه بسیار مهم است. این پارامترهای اولیه نقطه شروع فرآیند بهینه‌سازی را تعیین می‌کنند و بر سرعت و تاثیرگذاری شبکه می‌توانند به یک راه‌حل خوب همگرا شوند. وزن‌های اولیه بد انتخاب شده می‌تواند منجر به همگرایی کند، گیر کردن در مینیمم محلی یا ناتوانی در یادگیری کلی شود. برعکس، وزن‌های اولیه به خوبی انتخاب شده می‌توانند همگرایی سریع‌تر را تسهیل کرده و شبکه را قادر می‌سازند تا به حداقل‌های عمیق‌تر و مؤثرتر در فضای راه‌حل دست یابد. 
            در مورد شبکه‌های عصبی ضربه ای نیز، انتخاب وزن‌های اولیه یک عامل حیاتی است که می‌تواند تأثیر قابل‌توجهی بر تاثیر الگوریتم‌های یادگیری داشته باشد. این تنظیم وزن های اولیه زمینه را برای چگونگی سازگاری و عملکرد شبکه در طول فرآیند یادگیری فراهم می کند.

            ما در این آزمایش، وزن های اولیه مدل را با سه مقدار کم، نرمال و زیاد آزمایش و سپس نتایج را تحلیل میکنیم. مطابق شکل 
            \ref{fig:part2-images-stdp-different-weight}
            ملاحظه میکنیم که انتخاب وزن های خیلی کم یا خیلی زیاد هر دو میتوانند فرایند یادگیری را مختل کنند. به طور کلی، وزن های سیناپسی بیشتر، نرخ ضربه زدن نورون ها را بیشتر میکند و این وزن ها باید به گونه ای انتخاب شوند که نه خیلی کم باشند که از ضربه زدن جلوگیری شود و نه خیلی زیاد باشند که نورون ها به طور مداوم ضربه بزنند.
            \begin{figure}[htbp]
                \centering
                \includegraphics[width=1\textwidth]{plots/part2-images-stdp-different-weight.pdf} 
                \caption{قانون یادگیری 
                $STDP$ 
                با مقادیر مختلف وزن های اولیه.
                مطابق شکل مشاهده میکنیم که انتخاب وزن های خیلی کم ممکن است باعث شود کلا ضربه ای در جمعیت زده نشود و در نتیجه وزن ها نیز ثابت مانده و درکل فعالیتی دیده نشود. همچنین انتخاب وزن های خیلی زیاد نیز باعث می شود که نرخ ضربه زدن به شدت بالا رفته و تاثیر یادگیری نیز از بین برود و نورون ها فقط هنگام دریافت ورودی ضربه بزنند.
                }
                \label{fig:part2-images-stdp-different-weight}
            \end{figure}

        \subsubsection*{پارامتر های $\tau_{pre}$ و $\tau_{post}$}
            حال نوبت به آزمایش دو پارامتر 
            $\tau_{pre}$ و $\tau_{post}$ 
            می‌رسد. به طور کلی این دو پارامتر، میزان ردپای باقی مانده ضربه های پیش سیناپسی یا پس سیناپسی را تنظیم میکنند. علاوه بر مقدار خود این پارامتر ها، نسبت این دو پارامتر به یکدیگر نیز میتواند در یادگیری مدل موثر باشد. من در آزمایش هایی که داشتم، به طور کلی این نتیجه را گرفتم که این دو پارامتر باید تقریبا با هم برابر باشند ولی میزان پارامتر 
            $tau_{pre}$ 
            باید کمی بیشتر باشد.
            در این بخش آزمایش را صرفا برای نسبت های مختلف این دو پارامتر به یکدیگر بررسی میکنیم.

            همانطور که در شکل 
            \ref{fig:part2-images-stdp-different-tau}
            نیز ملاحظه میکنیم، خیلی کمتر بودن یا بیشتر نسبت 
            $tau_{pre}$ 
            به 
            $tau_{post}$ 
            می تواند باعث شود که مدل به خوبی نتواند الگو های ورودی را یاد بگیرد.
            \begin{figure}[htbp]
                \centering
                \includegraphics[width=1\textwidth]{plots/part2-images-stdp-different-tau.pdf} 
                \caption{قانون یادگیری 
                $STDP$ 
                با نسبت های مختلف پارامتر های 
                $\tau_{pre}$ و $\tau_{post}$.
                مطابق شکل بالا دریافت می‌شود که کمتر بودن نسبت پارامتر 
                $tau_{pre}$ 
                به 
                $tau_{post}$ 
                میتواند باعث شود که تاثیر ضربه های نورون های پس سیناپسی  به سرعت از بین برود و فقط ضربه های نورون های پیش سیناپسی روی وزن ها اثر بگذارند. در این حالت واضح است که یادگیری نمی تواند به خوبی انجام شود و مدل ها همزمان روی هر دو الگو ضربه می زنند. در حالت دیگر که نسبت 
                $tau_{post}$ 
                به 
                $tau_{pre}$ 
                بیشتر می شود، باعث می شود تاثیر ضربه های نورون های پیش سیناپسی تا مدت زیادی بماند طوری که حتی پس از ورودی الگوی جدید نیز ممکن است این ردپا ها حضور داشته باشند. در این حالت نیز واضح است که یادگیری به خوبی انجام نخواهد شد. به طور کلی این نتیجه را گرفتم که این دو پارامتر باید تقریبا با هم برابر باشند ولی میزان پارامتر 
                $tau_{pre}$ 
                باید کمی بیشتر باشد.
                }
                \label{fig:part2-images-stdp-different-tau}
            \end{figure}

        \clearpage
        \subsection{اضافه کردن دو نورون غیر فعال}
            در این قسمت به هر یک از دو لایه‌ی ورودی و خروجی یک نوون اضافه میکنیم که در طول آموزش ضربه‌ای نزنند و سپس وزن های متصل به این دو نورون را تحلیل میکنیم. از آنجا که توضیحات زیادی در این باره در فایل پروژه داده نشده است، ما فرض را بر این میگیریم که ضربه نزدن نورون خروجی، با صفر کردن وزن های متصل به آن اتفاق بیوفتد. روش دیگر برای اینکار، بالا بردن آستانه پتانسیل عمل آن نورون است. من هردو روش را آزمایش میکنم و نتایج را می آورم. برای روش اول،  که جلوگیری از ضربه زدن نورون توسط صفر کردن وزن سیناپسی اتفاق می افتد، مطابق شکل 
            \ref{fig:part2-stdp-two-additional-neuron-zero-weight}
            ملاحظه میکنیم که اضافه کردن یک نورون به لایه ورودی و یک نورون به لایه خروجی که ضربه ای نزند، باعث می شود که یادگیری به طور کامل با اختلال رو به رو شود و عملا یادگیری اتفاق نیوفتد.
            \begin{figure}[htbp]
                \centering
                \begin{subfigure}[b]{\linewidth}
                    \centering
                    \captionsetup{width=.9\linewidth}
                    \includegraphics[width=0.8\textwidth]{plots/part2-stdp-two-additional-neuron-zero-weight.pdf} 
                    \caption{جلوگیری از ضربه زدن نورون های اضافه شده از طریق صفر کردن وزن ها. مشاهده میکنیم که اضافه کردن نورون به ۲ لایه که ضربه ای نمیزنند، میتواند در یادگیری مدل اختلال ایجاد کند به طوری که دیگر یادگیری انجام نشود. همچنین در نمودار وزن ها ملاحظه میکنیم که یکی از وزن های متصل به نورون اضافه شده در ورودی و همچنین تمام وزن های نورون اضافه شده در خروجی تغییری نداشته اند.
                    }
                    \label{fig:part2-stdp-two-additional-neuron-zero-weight}
                \end{subfigure}
                \begin{subfigure}[b]{\linewidth}
                    \centering
                    \captionsetup{width=.9\linewidth}
                    \includegraphics[width=0.8\linewidth]{plots/part2-stdp-two-additional-neuron-zero-threshold.pdf} 
                    \caption{جلوگیری از ضربه زدن نورون های اضافه شده از طریق افزایش آستانه پتانسیل عمل نورون اضافه شده. مشاهده میکنیم، مانند شکل قبل اضافه کردن نورون به ۲ لایه، میتواند در یادگیری مدل اختلال ایجاد کند به طوری که عملا دیگر یادگیری انجام نمی‌شود. همچنین در نمودار وزن ها ملاحظه میکنیم که یکی از وزن های متصل به نورون اضافه شده در ورودی و همچنین تمام وزن های نورون اضافه شده در خروجی تغییری نداشته اند. تنها تفاوت با شکل قبل این است که وزن های نورون اضافه شده در خروجی صفر نیستند و همان مقادیر اولیه را دارند.
                    }
                    \label{fig:part2-stdp-two-additional-neuron-zero-threshold}
                \end{subfigure}
                \label{fig:part2-rstdp-two-additional-neuron}
            \end{figure}

        حال اگر وزن ها را نرمال سازی نکنیم، نمودار ها به صورت شکل 
        \ref{fig:part2-stdp-two-additional-neuron-zero-threshold-norm-off}
        تغییر خواهند کرد و یادگیری به طور کامل مختل خواهد شد.
        \begin{figure}[H]
            \centering
            \includegraphics[width=0.89\textwidth]{plots/part2-stdp-two-additional-neuron-zero-threshold-norm-off.pdf} 
            \caption{
                جلوگیری از ضربه زدن نورون های اضافه شده از طریق افزایش آستانه پتانسیل عمل نورون اضافه شده و نرمال سازی خاموش. مشاهده میکنیم، همانند شکل های 
                \ref{fig:part2-stdp-two-additional-neuron-zero-weight} 
                و 
                \ref{fig:part2-stdp-two-additional-neuron-zero-threshold} 
                با خاموش کردن نرمال سازی، یادگیری مدل به طور کامل مختل می شود. همچنین تغییر دیگری که ملاحظه می شود، تغییر در وزن های دیگر متصل به نورون جدید در ورودی است به طوری که وزن ها تا پایان شبیه سازی ثابت مانده اند.
            }
            \label{fig:part2-stdp-two-additional-neuron-zero-threshold-norm-off}
        \end{figure}

        \subsection{بررسی مدل با فعالیت کمینه برای لایه ها}
        در آخرین آزمایش مربوط به این قسمت، به بررسی رفتار مدل هنگامی که یک فعالیت کمینه در دو لایه وجود دارد را بررسی میکنیم. برای اینکار، آزمایش را با سه مقدار متفاوت فعالیت زمینه انجام میدهیم. همچنین حالت پایه آزمایش را، الگوی تصاویر با هم پوشانی نسبی در نظر میگیریم.
        همانطور که از شکل
        \ref{fig:part2-stdp-images-background-activity-low}
        دریافت می شود، به طور کلی افزودن یک جریان ثابت نویزی با، مانع یادگیری مدل نمی شود. 

        \begin{figure}[!ht]
            \centering
            \includegraphics[width=0.89\textwidth]{plots/part2-stdp-images-background-activity-low.pdf} 
            \caption{مدل با فعالیت زمینه کم.}
            \label{fig:part2-stdp-images-background-activity-low}
        \end{figure}
        \begin{figure}[!ht]
            \centering
            \includegraphics[width=0.89\textwidth]{plots/part2-stdp-images-background-activity-norm.pdf}
            \caption{مدل با فعالیت زمینه عادی.}
            \label{fig:part2-stdp-images-background-activity-norm}
        \end{figure}

        \begin{figure}[H]
            \centering
            \includegraphics[width=0.89\textwidth]{plots/part2-stdp-images-background-activity-high.pdf}
            \caption{مدل با فعالیت زمینه زیاد.
            مشاهده میکنیم که تغییر در میزان فعالیت زمینه، مشکلی در یادگیری مدل در زمان طولانی ایجاد نمیکند. داشتن فعالیت زمینه در اندازه عادی
            (شکل \ref{fig:part2-stdp-images-background-activity-norm})
            یعنی اندازه ای که نورون نیاز دارد تا برانگیخته شود، مشکلی برای فرایند یادگیری ایجاد نمیکند. اما زیاد از حد بودن فعالیت کمینه میتواند یادگیری را مدت زیادی به تاخیر بیندازد. در مقایسه با حالت قبل نیز میتوان گفت، که افزودن فعالیت کمینه دربرابر افزودن نورون بدون ضربه، کمتر یادگیری را مختل میکند.
            }
            \label{fig:part2-stdp-images-background-activity-high}
        \end{figure}

\newpage
\section{قانون یادگیری تقویتی}
    انعطاف‌پذیری وابسته به زمان ضربه تعدیل‌شده با پاداش 
    ($R-STDP$)\footnote{\lr{Reward-Modulated Spike-Timing-Dependent Plasticity}}
    به عنوان یک مدل پیچیده برای درک چگونگی تکامل ارتباطات سیناپسی بر اساس زمان‌بندی ضربه های عصبی و تأثیر سیگنال‌های پاداش عمل می‌کند. این مدل، که شامل تعدیل کننده های عصبی مانند دوپامین است، قدرت سیناپسی را برای ترویج رفتارهایی که منجر به پاداش می شود، تنظیم می کند. در اینجا، پیاده‌سازی 
    $R-STDP$ 
    را بررسی می‌کنیم که این پویایی‌ها را در یک چارچوب محاسباتی ادغام می‌کند و مطالعه مکانیسم‌های یادگیری را که زیربنای فرآیندهای یادگیری مبتنی بر پاداش است، تسهیل می‌کند.

    مدل 
    $R-STDP$ 
    که در اینجا پیاده سازی کرده ام، مکانیسم کلاسیک 
    $STDP$ 
    را با ترکیب اثرات دوپامین، که به عنوان یک سیگنال بازخورد برای تقویت یا مجازات اعمال عمل می کند، گسترش می دهد. این مدل برای رسیدگی به پیچیدگی‌های انعطاف‌پذیری سیناپسی تحت تأثیر زمان‌ ضربه ها و حضور سیگنال‌های پاداش، نوشته شده است.
    \begin{itemize}
        \item \textbf{ردیابی های سیناپسی و دوپامین:} 
        این مدل از اثر 
        (ردپا)
        های پیش سیناپسی و پس سیناپسی استفاده می کند که در طول زمان با ثابت های 
        $\tau_{pre}$ 
        و 
        $tau_{post}$ 
        کاهش می یابند. این ردپا ها زمان ضربه ها را ثبت می کنند و به عنوان یک متغیر اصلی برای محاسبه تغییرات سیناپسی عمل می کنند.
        \item \textbf{به‌روزرسانی‌ وزن ها با دوپامین:} 
        به‌روزرسانی‌های اثربخشی سیناپسی به تعامل بین آثار ضربه و سطوح دوپامین بستگی دارد، که به معنای این است که رفتار ها منجر به افزایش ضربه شده است یا خیر. این مدل به صورت پویا وزن ها را بر اساس اینکه آیا رفتار ها با نتایج مثبت 
        (دوپامین مثبت) 
        یا نتایج منفی 
        (دوپامین منفی) 
        مرتبط است، تنظیم می کند.
        \item \textbf{پاسخ‌های متفاوت دوپامین:} 
        این مدل بین سیگنال‌های دوپامین مثبت و منفی، تمایز قائل می‌شود که وزن‌های سیناپسی را بر این اساس تعدیل می‌کنند. این تنظیم دوگانه به مدل اجازه می دهد تا نقاط قوت سیناپسی را بر اساس نتایج خاص اقدامات تنظیم کند.
    \end{itemize}
    پاداش دادن به وزن ها نیز میتواند با روش های مختلفی انجام پذیرد. به عنوان مثال، ورودی نوع 
    $A$ 
    به شبکه ورودی داده می شود و انتظار داریم نورون خروجی شماره ۱ آن را تشخیص دهد. این تشخیص دادن را می توان به دو صورت در نظر گرفت. یک روش آن که نورونی که زودتر ضربه می زند، این تشخیص را انجام داده است، روش دیگر آنکه نورونی که در پنجره زمانی ورودی دادن بیشترین نرخ ضربه را داشته است، آن را تشخیص داده است. بسته به اینکه کدام روش انتخاب شود، به وزن های هر یک از نورون ها میتوانیم دوپامین ها را اعمال کنیم. من بدلیل آنکه ورودی ها نیز به صورت نرخ ضربه زدن کدگذاری شده بودند، از روش دوم برای تشخیص دادن نیز استفاده کردم.

    \subsection{آزمایش ها}
        \paragraph*{شباهت کسینوسی:}
                \textbf{ دقت شود که نمودار شباهت کسینوسی در تمام شکل ها آمده است.}

        حال به سراع آزمایش کردن مدلمان با ورودی و پارامتر های مختلف می رویم. اولین آزمایشی که مطابق معمول انجام میدهیم، ورودی دادن اندازه های مختلف ورودی است.
        \subsubsection*{تاثیر اندازه}
        اولین ورودی، ۲ الگوی ورودی از جنس یک آرایه از اعداد است. همانطور که در شکل 
        \ref{fig:part3-rstdp-small-array}
        نیز مشاهده می شود، مدل به خوبی توانسته است الگو های ورودی را یاد بگیرد.
        ورودی در این مدل، شامل دو آرایه تصادفی است که مقادیر یکی، دو برابر دیگری ست. مقادیر 
        \begin{figure}[!ht]
            \centering
            \captionsetup{width=.9\linewidth}
            \includegraphics[width=0.9\textwidth]{plots/part3-rstdp-small-array.pdf} 
            \caption{همانطور که در شکل مشاهده میکنیم، مدل پس از حدود ۵۰۰ مرحله توانسته است الگو های ورودی را یاد بگیرد. با گذشتن مراحل بیشتر نیز، نورون ها ورودی ها را بیشتر یاد گرفته به طوری که فعالیت کلی آن ها در طول بازه ورودی خود، بیشتر می شود. 
            همانطور که هر نورون در لایه خروجی یاد می‌گیرد که به ویژگی‌های خاص مرتبط با یک کلاس خاص به شدت پاسخ دهد، بردارهای وزن آن‌ها از هم جدا می‌شوند. این واگرایی منجر به کاهش شباهت کسینوسی می‌شود، زیرا بردارهای وزنی با تطتبیق خود برای تشخیص ویژگی‌های متمایز کلاس‌های مختلف، موازی‌تر می‌شوند. }
            \label{fig:part3-rstdp-small-array}
        \end{figure}
        
        حال دو الگوی 
        \ref{fig:part3-pattern1}
        و 
        \ref{fig:part3-pattern2} 
        را به عنوان ورودی به نورون ها ورودی میدهیم تا مشاهده کنیم که آیا شبکه از عهده یادگرفتن دو الگوی ساده برمی آید یا خیر.
        \begin{figure}[!ht]
            \centering
            \captionsetup{width=.9\linewidth}
            \begin{subfigure}[b]{0.35\textwidth}
                \centering
                \includegraphics[width=\textwidth]{images/pattern1.png}
                \caption{الگوی اول}
                \label{fig:part3-pattern1}
            \end{subfigure}
            \hfill
            \begin{subfigure}[b]{0.35\textwidth}
                \centering
                \includegraphics[width=\textwidth]{images/pattern2.png}
                \caption{الگوی دوم}
                \label{fig:part3-pattern2}
            \end{subfigure}
            \caption{دو تصویر به عنوان محرک}
            \label{fig:part3-patterns}
        \end{figure}
        همانطور که از شکل
        \ref{fig:part3-rstdp-small-patterns} 
        نیز بر می آید، نورون ها در یادگیری الگو های ذکر شده موفق بوده اند.

        \begin{figure}[!ht]
            \centering
            \captionsetup{width=.9\linewidth}
            \includegraphics[width=0.9\textwidth]{plots/part3-rstdp-small-patterns.pdf} 
            \caption{قانون یادگیری 
            $RSTDP$ برای دو الگوی ساده. مطابق نمودار بالا دریافت می شود که نورون ها با استفاده از قانون یادگیری 
            $RSTDP$ 
            توانسته اند الگو های 
            \ref{fig:part3-pattern1}
            و 
            \ref{fig:part3-pattern2} 
            را به خوبی یاد بگیرند. به طوری که این یادگیری تا ۵۰۰ مرحله یا به عبارتی با ۳ بار ورودی دادن هر الگو، انجام شده است که نکته قابل توجهی است. تا مرحله ۱۵۰۰ ام شبیه سازی نیز این یادگیری عمیق تر شده و نورون ها  فقط در زمانی که الگوی خود را میبینند ضربه می زنند.
            }
            \label{fig:part3-rstdp-small-patterns}
        \end{figure}
        حال ورودی ها را به طوری که در صورت پروژه گفته شده بود به شبکه میدهیم، یعنی بعضی نورون ها، هر دو الگو را به عنوان ورودی دریافت میکنند.(نورون های سبز در شکل
        \ref{fig:part3-input-overlap})

        \begin{figure}[!ht]
            \centering
            \captionsetup{width=.9\linewidth}
            \includegraphics[width=0.9\textwidth]{images/input-overlap.png} 
            \caption{نورون هایی که با رنگ سبز مشخص شده اند، هر دو الگو را به عنوان ورودی دریافت میکنند.
            }
            \label{fig:part3-input-overlap}
        \end{figure}

        مطابق شکل 
        \ref{fig:part3-rstdp-small-patterns-overlap}
        مشاهده میکنیم که هم پوشانی نورون های ورودی در دریافت محرک، باعث می شود که شبکه کمی دیرتر الگو ها را یاد بگیرند.
        \begin{figure}[!ht]
            \centering
            \captionsetup{width=.9\linewidth}
            \includegraphics[width=0.9\textwidth]{plots/part3-rstdp-small-patterns-overlap.pdf} 
            \caption{هم پوشانی نورون ها در دریافت ورودی. مطابق نمودار های بالا، اختصاص بعضی نورون ها به دریافت دو نوع الگو، باعث می شود که شبکه کمی دیرتر الگو ها را یاد بگیرد. به طوری که در مقایسه با شکل
            \ref{fig:part3-rstdp-small-patterns}
            یادگیری در ۳۰۰ مرحله بعد انجام شده است. هر چند هر دو مدل در نهایت الگو های ورودی را فراگرفته اند.
            }
            \label{fig:part3-rstdp-small-patterns-overlap}
        \end{figure}
        حتی اگر میزان همپوشانی را به حداکثر برسانیم
        (به طوری که همه محرک ها به همه نورون ها داده شود)
        باز هم شبکه قادر به تشخیص الگو ها خواهد بود.
        (شکل \ref{fig:part3-rstdp-small-patterns-high-overlap})

        \begin{figure}[!ht]
            \centering
            \captionsetup{width=.9\linewidth}
            \includegraphics[width=0.9\textwidth]{plots/part3-rstdp-small-patterns-high-overlap.pdf} 
            \caption{هم پوشانی حداکثری نورون ها در دریافت ورودی. مشاهده میکنیم که حتی با هم پوشانی کامل نیز، شبکه قادر به یادگیری الگو ها می باشد، هر چند بعضی اوقات نورون ها هنگامی که ورودی دیگر را میبینند نیز ضربه میزنند که اینکار با تنظیم پارامتر ها قابل بهبود است و در بخش بعدی به آن می پردازیم.
            }
            \label{fig:part3-rstdp-small-patterns-high-overlap}
        \end{figure}

        حال در نهایت یک قدم فراتر گذاشته، و دو تصویر با اندازه های 
        $10\times 10$ 
        را به عنوان ورودی به مدل میدهیم.
        (تصاویر 
        \ref{fig:part3-input-image1} و
        \ref{fig:part3-input-image2})

        \begin{figure}[htbp]
            \centering
            \captionsetup{width=.9\linewidth}
            \begin{subfigure}[b]{0.35\textwidth}
                \centering
                \includegraphics[width=\textwidth]{images/slop-resized.jpg}
                \caption{تصویر اول}
                \label{fig:part3-input-image1}
            \end{subfigure}
            \hfill
            \begin{subfigure}[b]{0.35\textwidth}
                \centering
                \includegraphics[width=\textwidth]{images/bird-resized.jpg}
                \caption{تصویر دوم}
                \label{fig:part3-input-image2}
            \end{subfigure}
            \caption{یک تصویر به عنوان محرک}
            \label{fig:part3-input-images}
        \end{figure}
        ممکن است انتظار داشته باشیم با زیاد شدن ابعاد ورودی، تشخیص الگوها نیز سخت تر شود، اما شکل 
        \ref{fig:part3-rstdp-images}
        این فرض را رد میکند و مشاهده میکنیم که یادگیری دو الگو به خوبی انجام شده است.

        \begin{figure}[!ht]
            \centering
            \captionsetup{width=.9\linewidth}
            \includegraphics[width=0.9\textwidth]{plots/part3-rstdp-images.pdf} 
            \caption{یادگیری الگوی دو تصویر توسط قانون یادگیری 
            $RSTDP$. مطابق شکل بالا مشاهده میکنیم که با وجود ابعاد بالای الگوی ورودی نیز، یادگیری هنوز به خوبی انجام می شود. نکته قابل توجه دیگری که از این شکل برداشت می شود نیز این است که پس از حدود ۱۵۰۰ مرحله از شبیه سازی، نمودار فعالیت نورون ها ثابت تر می شود، بدان معنا که بین تکرار های ۱۰۰۰ و ۱۵۰۰، فعالیت نورون ها در بعضی لحظه ها 
            $50\%$ 
            ودر بعضی لحظه ها کمتر است اما بعد از تکرار ۱۵۰۰، هر گاه که ورودی مربوط به یک الگو داده می شود، در تمام آن پنجره زمانی، نورون های مربوط ضربه میزنند و پیوسته فعالیت کامل دارند.
            }
            \label{fig:part3-rstdp-images}
        \end{figure}

        مدل نسبت به هم پوشانی نورون ها برای گرفتن ورودی نیز پایدار است و همچنان الگو ها را یاد میگیرد.
        (شکل \ref{fig:part3-rstdp-images-overlap})
        \begin{figure}[!ht]
            \centering
            \captionsetup{width=.9\linewidth}
            \includegraphics[width=0.9\textwidth]{plots/part3-rstdp-images-overlap.pdf} 
            \caption{یادگیری الگوی دو تصویر توسط قانون یادگیری 
            $RSTDP$ با هم پوشانی نورون ها. مطابق شکل ملاحظه میکنیم که هم پوشانی نورون ها در دریافت ورودی نیز نمیتواند از یادگیری الگو ها توسط مدل جلو گیری کند. هرچند این اتفاق کمی دیرتر می افتد که به این دلیل است بعضی نورون ها باید هر دوی الگو ها را یاد بگیرند.
            (توجه کنید که اکنون جمعیت ورودی شامل ۱۵۰ نورون است درحالی که الگو ها ۱۰۰ پیکسل دارند و از آنجا که بیشترین میزان فعالیت نورون ها بعد از یادگیری 
            $50\%$ 
            است، ۷۵ نورون باید ۱۰۰ پیکسل را فراگیرند. به عبارتی دیگر، بعضی نورون ها هر دو الگو را یاد میگیرند.)
            }
            \label{fig:part3-rstdp-images-overlap}
        \end{figure}
        حتی با بیشتر کردن هم پوشانی نیز باز هم مدل می تواند تصاویر را یاد بگیرد.
        (شکل 
        \ref{fig:part3-rstdp-images-high-overlap})
        هر چند نسبت به حالت های قبلی این یادگیری دیرتر اتفاق می افتد، اما با تنظیم پارامتر هایی مانند پنجره زمانی یا دوپامین که در ادامه بررسی می شوند می توان این یادگیری را سریع تر کرد.

        \begin{figure}[!ht]
            \centering
            \captionsetup{width=.9\linewidth}
            \includegraphics[width=0.9\textwidth]{plots/part3-rstdp-images-high-overlap.pdf} 
            \caption{یادگیری الگوی دو تصویر توسط قانون یادگیری 
            $RSTDP$ با هم پوشانی کامل نورون ها.
            همانطور که در نمودار بالا نیز ملاحظه می شود، حتی با هم پوشانی کامل نورون ها نیز دو الگو از یکدیگر تشخیص داده می شوند. نکته ایکه وجود دارد این است که این یادگیری نسبت با حالت های با هم پوشانی کمتر، دیرتر اتفاق می افتد.
            }
            \label{fig:part3-rstdp-images-high-overlap}
        \end{figure}

        
        \clearpage
        \subsubsection*{تاثیر وزن های اولیه}
            در گذشته در شبکه های عصبی کلاسیک، وزن های اولیه ممکن بود تاثیر زیادی در یادگیری شبکه بگذارند. در این بخش ما یادگیری مدلمان را روی دو الگوی ورودی، برای وزن های متفاوت آزمایش میکنیم. بدلیل زیاد بودن نمودار ها، فقط نمودار های متفاوت و مهم را در این قسمت می آوریم.
            مطابق شکل 
            \ref{fig:part3-rstdp-images-different-weights}
            مشاهده میکنیم که مقادیر وزن های اولیه می تواند در روند یادگیری مدل تاثیر بگذارد.
            \begin{figure}[!ht]
                \centering
                \captionsetup{width=.9\linewidth}
                \includegraphics[width=1\textwidth]{plots/part3-rstdp-images-different-weights.pdf} 
                \caption{تاثیر وزن های اولیه بر یادگیری با قانون 
                $RSTDP$. 
                مطابق نمودار بالا مشاهده میکنیم که وزن های اولیه میتواند بر یادگیری اولیه مدل تاثیر بگذارد. به طور که انتخاب وزن های کم باعث می شود که در طول شبیه سازی نورون ها ضربه ای نزنند و یادگیری اتفاق نیوفتد، و انتخاب وزن های اولیه بزرگ نیز ممکن است باعث شود که نرخ ضربه زدن نورون ها بیش از حد شود و یادگیری به درستی انجام نشود.
                }
                \label{fig:part3-rstdp-images-different-weights}
            \end{figure}

        \newpage
        \subsubsection*{تاثیر دوپامین}
            میتوان گفت دوپامین مهم ترین تفاوت 
            $RSTDP$ 
            با 
            $STDP$ 
            است. از این رو انتظار داریم که تغییر در این پارامتر بتواند رفتار مدل را نیز تغییر دهد.
            مطابق شکل 
            \ref{fig:part3-rstdp-images-different-dopamine}
            مشاهده میکنیم که تغییر در میزان دوپامین، میتواند وزن های سیناپسی و سرعت یادگیری را تغییر دهد. هر چند در هر سه مدل، در نهایت یادگیری انجام شده است.

            \begin{figure}[!ht]
                \centering
                \captionsetup{width=.9\linewidth}
                \includegraphics[width=0.9\textwidth]{plots/part3-rstdp-images-different-dopamine.pdf} 
                \caption{تاثیر میزان دوپامین بر یادگیری با قانون 
                $RSTDP$. 
                مطابق نمودار بالا مشاهده میکنیم که تغییر در میزان دوپامین می تواند سرعت یادگیری را تغییر دهد. به عنوان مثال، افزایش دوپامین توانسته است به ترتیب از چپ به راست یادگیری را در مراحل زودتری تمام کند. نکته دیگر این است که افزایش دوپامین، باعث شده است که بعضی وزن های سیناپسی نیز از حد خود فراتر رفته و بسیار بزرگ شوند.
                }
                \label{fig:part3-rstdp-images-different-dopamine}
            \end{figure}
        
        حال که تاثیر انواع پارامتر های موثر در مدل را بررسی کردیم میتوانیم یک مدل خوب برای یادگیری دو الگوی متفاوت 
        (تصاویر 
        \ref{fig:part3-input-image1} و
        \ref{fig:part3-input-image2}) 
        را به عنوان ورودی به مدل بدهیم تا به خوبی آن ها را یاد بگیرد. این مدل در شکل 
        \ref{fig:part3-rstdp-images-high-overlap-adjusted}
        آمده است.
        \begin{figure}[!ht]
            \centering
            \captionsetup{width=.9\linewidth}
            \includegraphics[width=0.9\textwidth]{plots/part3-rstdp-images-high-overlap-adjusted.pdf} 
            \caption{یادگیری صحیح دو الگوی تصویر توسط قانون یادگیری 
            $RSTDP$. همانطور که ملاحظه می شود، پس از حدود ۵۰۰ مرحله شبیه سازی مدل به خوبی الگو ها را یادگرفته است.
            }
            \label{fig:part3-rstdp-images-high-overlap-adjusted}
        \end{figure}


        % \paragraph*{تاثیر $\tau_{pre}$ و $\tau_post$}
        %     حال تاثیر پارمتر های 
    \newpage
    \subsection{اضافه کردن دو نورون غیر فعال}
        در این قسمت به هر یک از دو لایه‌ی ورودی و خروجی یک نوون اضافه میکنیم که در طول آموزش ضربه‌ای نزنند و سپس وزن های متصل به این دو نورون را تحلیل میکنیم. از آنجا که توضیحات زیادی در این باره در فایل پروژه داده نشده است، ما فرض را بر این میگیریم که ضربه نزدن نورون خروجی، با صفر کردن وزن های متصل به آن اتفاق بیوفتد. روش دیگر برای اینکار، بالا بردن آستانه پتانسیل عمل آن نورون است. من هردو روش را آزمایش میکنم و نتایج را می آورم. برای روش اول،  که جلوگیری از ضربه زدن نورون توسط صفر کردن وزن سیناپسی اتفاق می افتد، مطابق شکل 
        \ref{fig:part3-rstdp-two-additional-neuron-zero-weight}
        ملاحظه میکنیم که اضافه کردن یک نورون به لایه ورودی و یک نورون به لایه خروجی که ضربه ای نزند، باعث می شود که یادگیری کمی با اختلال مواجه شود، هر چند هنوز دو نورون قبلی خروجی قادر به تشخیص الگو ها هستند، ولی فعالیت آن ها هنگام تشخیص دادن الگو های کمی کاهش یافته است.

        حال جلوگیری از ضربه زدن نورون های اضافه شده را با استفاده از افزایش آستانه پتانسیل عمل انجام میدهیم. مجددا مانند حالت قبل ملاحظه میکنیم که یادگیری با اختلال مواجه شده است، هر چند هنوز الگو ها تشخیص داده می شوند.
        (شکل 
        \ref{fig:part3-rstdp-two-additional-neuron-zero-threshold})
        \begin{figure}[htbp]
            \centering
            \begin{subfigure}[b]{\linewidth}
                \centering
                \captionsetup{width=.9\linewidth}
                \includegraphics[width=0.8\textwidth]{plots/part3-rstdp-two-additional-neuron-zero-weight.pdf} 
                \caption{جلوگیری از ضربه زدن نورون های اضافه شده از طریق صفر کردن وزن ها. مشاهده میکنیم که اضافه کردن نورون به ۲ لایه که ضربه ای نمیزنند، میتواند در یادگیری مدل اختلال ایجاد کند. هر چند هنوز مدل می تواند تا حدی الگوها را تشخیص دهد. همچنین در نمودار وزن ها ملاحظه میکنیم که یکی از وزن های متصل به نورون اضافه شده در ورودی و همچنین تمام وزن های نورون اضافه شده در خروجی تغییری نداشته اند.
                }
                \label{fig:part3-rstdp-two-additional-neuron-zero-weight}
            \end{subfigure}
            \begin{subfigure}[b]{\linewidth}
                \centering
                \captionsetup{width=.9\linewidth}
                \includegraphics[width=0.8\linewidth]{plots/part3-rstdp-two-additional-neuron-zero-threshold.pdf} 
                \caption{جلوگیری از ضربه زدن نورون های اضافه شده از طریق افزایش آستانه پتانسیل عمل نورون اضافه شده. مشاهده میکنیم، مانند شکل قبل اضافه کردن نورون به ۲ لایه، میتواند در یادگیری مدل اختلال ایجاد کند. هر چند هنوز مدل می تواند تا حدی الگوها را تشخیص دهد. همچنین در نمودار وزن ها ملاحظه میکنیم که یکی از وزن های متصل به نورون اضافه شده در ورودی و همچنین تمام وزن های نورون اضافه شده در خروجی تغییری نداشته اند. تنها تفاوت با شکل قبل این است که وزن های نورون اضافه شده در خروجی صفر نیستند و همان مقادیر اولیه را دارند.
                }
                \label{fig:part3-rstdp-two-additional-neuron-zero-threshold}
            \end{subfigure}
            \label{fig:part3-rstdp-two-additional-neuron}
        \end{figure}

        حال اگر وزن ها را نرمال سازی نکنیم، نمودار ها به صورت شکل 
        \ref{fig:part3-rstdp-two-additional-neuron-zero-threshold-norm-off}
        تغییر خواهند کرد و یادگیری به طور کامل مختل خواهد شد.
        \begin{figure}[H]
            \centering
            \includegraphics[width=0.89\textwidth]{plots/part3-rstdp-two-additional-neuron-zero-threshold-norm-off.pdf} 
            \caption{
                جلوگیری از ضربه زدن نورون های اضافه شده از طریق افزایش آستانه پتانسیل عمل نورون اضافه شده و نرمال سازی خاموش. مشاهده میکنیم، برخلاف شکل های 
                \ref{fig:part3-rstdp-two-additional-neuron-zero-weight} 
                و 
                \ref{fig:part3-rstdp-two-additional-neuron-zero-threshold} 
                با خاموش کردن نرمال سازی، یادگیری مدل به طور کامل مختل می شود.
            }
            \label{fig:part3-rstdp-two-additional-neuron-zero-threshold-norm-off}
        \end{figure}
    \clearpage
    \subsection{بررسی مدل با فعالیت کمینه برای لایه ها}
        در آخرین آزمایش مربوط به این قسمت، به بررسی رفتار مدل هنگامی که یک فعالیت کمینه در دو لایه وجود دارد را بررسی میکنیم. برای اینکار، آزمایش را با سه مقدار متفاوت فعالیت زمینه انجام میدهیم. همچنین حالت پایه آزمایش را، الگوی تصاویر با هم پوشانی نسبی در نظر میگیریم.
        همانطور که از شکل
        \ref{fig:part3-rstdp-images-background-activity-low}
        دریافت می شود، به طور کلی افزودن یک جریان ثابت نویزی با، مانع یادگیری مدل نمی شود. حتی با انتخاب معقولی از فعالیت زمینه، ممکن است یادگیری مدل بهتر نیز بشود،
        (شکل \ref{fig:part3-rstdp-images-background-activity-norm})

        \begin{figure}[!ht]
            \centering
            \includegraphics[width=0.89\textwidth]{plots/part3-rstdp-images-background-activity-low.pdf} 
            \caption{مدل با فعالیت زمینه کم.}
            \label{fig:part3-rstdp-images-background-activity-low}
        \end{figure}
        \begin{figure}[!ht]
            \centering
            \includegraphics[width=0.89\textwidth]{plots/part3-rstdp-images-background-activity-norm.pdf}
            \caption{مدل با فعالیت زمینه عادی.}
            \label{fig:part3-rstdp-images-background-activity-norm}
        \end{figure}
        \begin{figure}[H]
            \centering
            \includegraphics[width=0.89\textwidth]{plots/part3-rstdp-images-background-activity-high.pdf}
            \caption{مدل با فعالیت زمینه زیاد.
            مشاهده میکنیم که تغییر در میزان فعالیت زمینه، مشکلی در یادگیری مدل در زمان طولانی ایجاد نمیکند. هرچند که داشتن فعالیت زمینه در اندازه عادی
            (شکل \ref{fig:part3-rstdp-images-background-activity-norm})
            میتواند حتی باعث بهتر شدن یادگیری نیز بشود، اما زیاد از حد بودن فعالیت کمینه میتواند یادگیری را کمی به تاخیر بیندازد. در مقایسه با حالت قبل نیز میتوان گفت، که افزودن فعالیت کمینه دربرابر افزودن نورون بدون ضربه، کمتر یادگیری را مختل میکند.
            \label{fig:part3-rstdp-images-background-activity-high}}
        \end{figure}


%%%%%%%%%%%%%%%%%%%%%%%%%%% Bibliography section %%%%%%%%%%%%%%%%%%%%%%%%%%%
\begin{thebibliography}{1}
    \bibitem{textbook}
        \begin{latin}
            Computational Neuroscience Course, School of computer science, University of Tehran
        \end{latin}
    \bibitem{PymoNNtorch}
        \begin{latin}
            PymoNNtorchPytorch-adapted version of PymoNNto
        \end{latin}
    \bibitem{Neuronal-Dynamics}
        \begin{latin}
            \href{https://neuronaldynamics.epfl.ch/online/Ch12.S3.html}{Neuronal Dynamics, Wulfram Gerstner, Werner M. Kistler, Richard Naud and Liam Paninski
            }
        \end{latin}
    \bibitem{Poisson-Distribution-Wikipedia}
        \begin{latin}
            Poisson Distribution. Wikipedia
            [\href{https://en.wikipedia.org/wiki/Poisson_distribution}{Link}]
        \end{latin}
    \bibitem{STDP-Wikipedia}
        \begin{latin}
            Spike-timing-dependent plasticity. Wikipedia
            [\href{https://en.wikipedia.org/wiki/Spike-timing-dependent_plasticity}{Link}]
        \end{latin}
    \end{thebibliography}
\end{document}
