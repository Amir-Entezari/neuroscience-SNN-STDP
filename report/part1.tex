
\newpage
\section{پیاده سازی کدگذاری ها}
    در این بخش می‌خواهیم نحوه کد کردن اطلاعات
    (مانند یک تصویر) 
    در شبکه های عصبی ضربه‌ای را پیاده سازی و تحلیل کنیم. برای اینکار، روش های مختلفی وجود دارد، مانند روش 
    \lr{time-to-first-spike}، 
    کدگذاری ترتیبی
    \footnote{\lr{Rank-order encoding}}، 
    کدگذاری براساس تاخیر
    \footnote{\lr{Latency encoding}}، 
    کدگذاری براساس همزمانی ضربه ها
    \footnote{\lr{Coding by synchrony}}، 
    کدگذاری اعداد، کدگذاری به روش پواسون
    \footnote{Poisson} 
    و روش های دیگر بسیار که کاربرد های مختلفی دارند. 
    در این پروژه، ما تمرکزمان را روی سه روش کدگذاری زیر میگذاریم و آن ها پیاده سازی و تحلیل میکنیم:
    \begin{itemize}
        \item روش کدگذاری 
        \lr{time-to-first-spike}
        \item روش کدگذاری مقادیر عددی
        \item روش کدگذاری به کمک توزیع پواسون
    \end{itemize}

    \subsection*{کدگذاری به روش 
    \lr{time-to-first-spike}}
        